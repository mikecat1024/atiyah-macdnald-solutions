\documentclass[dvipdfmx]{jsarticle}

\usepackage{amsthm}
\usepackage{enumerate}
\usepackage{amsmath}
\usepackage{amssymb}
\usepackage{cleveref}
\usepackage{physics}
\usepackage{color}
\usepackage{tikz-cd}

\newcounter{BaseCounter}[section]

\newtheoremstyle{JapanesePropositionStyle}{24pt}{}{}{}{\bfseries}{.}{1em}{\thmname{#1}\hspace{0.2em}\thmnumber{#2}\thmnote{\hspace{0.5em}(#3)}}
\theoremstyle{JapanesePropositionStyle}

\newtheorem{proposition}[BaseCounter]{Proposition}
\newtheorem{problem}[BaseCounter]{Problem}
\newtheorem*{note}{Note}
\newtheorem{lemma}[BaseCounter]{Lemma}
\newtheorem{corollary}[BaseCounter]{Corollary}
\newtheorem{theorem}[BaseCounter]{Theorem}
\newtheorem{definition}[BaseCounter]{Definition}
\newtheorem{example}[BaseCounter]{Example}

\makeatletter
\newcommand\RedeclareMathOperator{%
  \@ifstar{\def\rmo@s{m}\rmo@redeclare}{\def\rmo@s{o}\rmo@redeclare}%
}
\newcommand\rmo@redeclare[2]{%
  \begingroup \escapechar\m@ne\xdef\@gtempa{{\string#1}}\endgroup
  \expandafter\@ifundefined\@gtempa
     {\@latex@error{\noexpand#1undefined}\@ehc}%
     \relax
  \expandafter\rmo@declmathop\rmo@s{#1}{#2}}
\newcommand\rmo@declmathop[3]{%
  \DeclareRobustCommand{#2}{\qopname\newmcodes@#1{#3}}%
}
\@onlypreamble\RedeclareMathOperator
\makeatother

\DeclareMathOperator{\ch}{char}
\DeclareMathOperator{\Hom}{Hom}
\DeclareMathOperator{\Emb}{Emb}
\DeclareMathOperator{\Aut}{Aut}
\DeclareMathOperator{\sign}{sign}
\DeclareMathOperator{\supp}{supp}
\DeclareMathOperator{\interior}{int}
\DeclareMathOperator{\relint}{relint}
\DeclareMathOperator{\Div}{Div}
\DeclareMathOperator{\Spec}{Spec}
\DeclareMathOperator{\Max}{Max}
\DeclareMathOperator{\coker}{coker}
\DeclareMathOperator{\Quot}{Quot}

\DeclareMathOperator*{\colim}{colim}

\RedeclareMathOperator{\Im}{Im}

\newcommand{\emb}{\hookrightarrow}

\newcommand{\Mod}[1]{\textbf{Mod}_{#1}}
\newcommand{\fMod}[1]{\textbf{fMod}_{#1}}
\newcommand{\QCoh}[1]{\textbf{QCoh}_{#1}}
\newcommand{\Coh}[1]{\textbf{Coh}_{#1}}

\renewcommand{\theenumi}{\arabic{enumi}}
\renewcommand{\labelenumi}{(\theenumi)}


\begin{document}
    \begin{problem}
        \begin{align*}
            \mbox{$P$は非特異点}
            &\Longleftrightarrow f \notin \mathfrak{m}^2\\
            &\Longleftrightarrow \dim \mathfrak{m}/\mathfrak{m}^2 = n-1\\
            &\Longleftrightarrow \mbox{$A$は正則局所環}
        \end{align*}
        となる.
        実際, ある極大イデアル$\mathfrak{m}_0 \in \Max{k[X_1, \dots, X_n]}$
        が存在して, $\mathfrak{m} = \mathfrak{m}_0/(f)$
        と表せるが, このとき,
        \[
            \mathfrak{m}/\mathfrak{m}^2 = (\mathfrak{m}_0/(f))/(\mathfrak{m}_0/(f))^2 =
            (\mathfrak{m}_0/(f))/(\mathfrak{m}_0^2/\mathfrak{m}_0^2 \cap (f)) = \mathfrak{m}_0^2/\mathfrak{m}_0^2 + (f)
        \]
        となる.
        Hilbertの零点定理より,
        $\mathfrak{m}_0 = (X_1, \dots, X_n)$と仮定しても一般性は失わない.
        このとき, $P = 0$となるので, $f(0) = 0$であり, $f \in \mathfrak{m}_0$であることに注意する.
        $A = k[X_1, \dots, X_n]/(f)$とおく.
        (10.18)より,
        \[
            n-1 = \dim{k[X_1, \dots, X_n]} - 1  = \dim A_\mathfrak{m} \leq \dim_k(\mathfrak{m}/\mathfrak{m}^2) = \dim_k \mathfrak{m}_0/\mathfrak{m}_0^2 + (f)
        \]
        であって,
        $f \notin \mathfrak{m}_0^2$ならば, $\dim_k \mathfrak{m}_0/\mathfrak{m}_0^2 + (f) < \dim_k \mathfrak{m}_0/\mathfrak{m}_0^2 = n$
        となるので, $\dim_k\mathfrak{m}_0/\mathfrak{m}_0^2 + (f) \leq n-1$が成り立つ.
        これより, $\dim_\mathfrak{m} A_\mathfrak{m} = \dim \mathfrak{m}/\mathfrak{m}^2$となって, $A$は正則局所環となる.

        逆に, $\dim_k \mathfrak{m}/\mathfrak{m}^2 = n-1$とすれば, 同様にして, $f \notin \mathfrak{m}_0^2$となるので, 従う.
    \end{problem}

    \begin{problem}
        $\mathfrak{m}$を$k[\![T_1, \dots, T_n]\!]$の極大イデアルとする.
        このとき, $k[\![T_1, \dots, T_n]\!]/\mathfrak{m}^n \to A/(x_1, \dots, x_n)^n$は単射であって,
        7.16より, ある$m$が存在して,
        \[
            \mathfrak{m}^m \subseteq (x_1, \dots, x_n) \subseteq \mathfrak{m}
        \]
        が成り立つ.
        これより, $\mathfrak{m}$進完備化と$(x_1, \dots, x_n)$進完備化は等しい.
        これと10.2より, $k[\![T_1, \dots, T_n]\!] \to A$は単射である.

        $\phi :k[\![T_1, \dots, T_n]\!] \to A$とする.
        \[
            \bigcap_i (x_1, \dots, x_n)^i = \phi^{-1}\qty(\bigcap_n \mathfrak{m}) = 0
        \]
        なので, $A$は$k[\![T_1, \dots, T_n]\!]$の$\mathfrak{m}$フィルター位相によってハウスドルフである.
        また,
        $G(A)$は有限生成$G(k[\![T_1, \dots, T_n]\!])$加群である.
        実際, $k[\![T_1, \dots, T_n]\!]$がネーターであることと, (10.22)より,
        $G_\mathfrak{m}(k[\![T_1, \dots, T_n]\!])$はNotherianである.
        したがって, $G(A)$はNotherian加群なので, 特に有限生成$G(k[\![T_1, \dots, T_n]\!])$加群である.
        これらと, 10.24より, $A$は有限生成$k[\![T_1, \dots, T_n]\!]$加群である.
    \end{problem}

    \begin{problem}

    \end{problem}

    \stepcounter{BaseCounter}

    \begin{problem}
        $\lambda$を加法関数とし, $\gamma(M)$で$(M)$の$K(A)$での像を表和すことにすれば,
        7.26より, ある$\lambda_0: K(A) \to \mathbb{Z}$が存在して,
        $\lambda(M) = \lambda_0(\gamma(M))$が成り立つので,
        Poincare級数は
        \[
            P(M, t) = \sum_{n = 0}^\infty \lambda_0(\gamma(M_n))t^n
         \]
         と表せる.
         これによって, (11.1)を書き換えればよい.
    \end{problem}

    \begin{problem}
        $\mathfrak{p}$を$A$における素イデアル鎖において最大の素イデアルとすれば,
        $\mathfrak{p} \subsetneq \mathfrak{p} + (x)$であって, $\mathfrak{p} + (x)$は素イデアルである.
        実際, 包含関係は明らかで, $A[X] \to A \to A/\mathfrak{p}$を考えれば,
        $\mathfrak{p} + (x)$は素イデアルなので, $1 + \dim{A} \leq \dim{A[X]}$が成り立つ.

        次に, $A \to A[X]$を埋め込みとして, $\Spec{A[X]} \to Spec{A}$を考える.
        $\mathfrak{p} \in \Spec{A}$に対して,
        ${f^\ast}^{-1}(\mathfrak{p}) \cong \Spec{k(\mathfrak{p}) \otimes_A A[X]} = \Spec{k(\mathfrak{p})[X]}$
        であって, $k(\mathfrak{p})[X]$はPIDなので,
        $\dim k(\mathfrak{p})[X] = 1$となる.
        したがって,
        $\dim{A} \leq 2 \dim{A} + 1$が成り立つ.
    \end{problem}

    \begin{problem}
        11.7より, $1 + \dim{A} \leq \dim{A[X]}$が成り立つので, 逆を示せばよい.
        \color{red}
            こんどやる
        \color{black}
    \end{problem}
\end{document}
