\documentclass[dvipdfmx]{jsarticle}

\usepackage{amsthm}
\usepackage{enumerate}
\usepackage{amsmath}
\usepackage{amssymb}
\usepackage{cleveref}
\usepackage{physics}
\usepackage{color}
\usepackage{tikz-cd}

\newcounter{BaseCounter}[section]

\newtheoremstyle{JapanesePropositionStyle}{24pt}{}{}{}{\bfseries}{.}{1em}{\thmname{#1}\hspace{0.2em}\thmnumber{#2}\thmnote{\hspace{0.5em}(#3)}}
\theoremstyle{JapanesePropositionStyle}

\newtheorem{proposition}[BaseCounter]{Proposition}
\newtheorem{problem}[BaseCounter]{Problem}
\newtheorem*{note}{Note}
\newtheorem{lemma}[BaseCounter]{Lemma}
\newtheorem{corollary}[BaseCounter]{Corollary}
\newtheorem{theorem}[BaseCounter]{Theorem}
\newtheorem{definition}[BaseCounter]{Definition}
\newtheorem{example}[BaseCounter]{Example}

\makeatletter
\newcommand\RedeclareMathOperator{%
  \@ifstar{\def\rmo@s{m}\rmo@redeclare}{\def\rmo@s{o}\rmo@redeclare}%
}
\newcommand\rmo@redeclare[2]{%
  \begingroup \escapechar\m@ne\xdef\@gtempa{{\string#1}}\endgroup
  \expandafter\@ifundefined\@gtempa
     {\@latex@error{\noexpand#1undefined}\@ehc}%
     \relax
  \expandafter\rmo@declmathop\rmo@s{#1}{#2}}
\newcommand\rmo@declmathop[3]{%
  \DeclareRobustCommand{#2}{\qopname\newmcodes@#1{#3}}%
}
\@onlypreamble\RedeclareMathOperator
\makeatother

\DeclareMathOperator{\ch}{char}
\DeclareMathOperator{\Hom}{Hom}
\DeclareMathOperator{\Emb}{Emb}
\DeclareMathOperator{\Aut}{Aut}
\DeclareMathOperator{\sign}{sign}
\DeclareMathOperator{\supp}{supp}
\DeclareMathOperator{\interior}{int}
\DeclareMathOperator{\relint}{relint}
\DeclareMathOperator{\Div}{Div}
\DeclareMathOperator{\Spec}{Spec}
\DeclareMathOperator{\Max}{Max}
\DeclareMathOperator{\coker}{coker}
\DeclareMathOperator{\Quot}{Quot}

\DeclareMathOperator*{\colim}{colim}

\RedeclareMathOperator{\Im}{Im}

\newcommand{\emb}{\hookrightarrow}

\newcommand{\Mod}[1]{\textbf{Mod}_{#1}}
\newcommand{\fMod}[1]{\textbf{fMod}_{#1}}
\newcommand{\QCoh}[1]{\textbf{QCoh}_{#1}}
\newcommand{\Coh}[1]{\textbf{Coh}_{#1}}

\renewcommand{\theenumi}{\arabic{enumi}}
\renewcommand{\labelenumi}{(\theenumi)}


\begin{document}
    \begin{problem}
        $V(J) \subseteq B$を閉集合として, $f^\ast(V(J)) = V(f^{-1}(J))$を示す.
        $\mathfrak{p} \in f^{\ast}(V(J))$とすれば, ある$\mathfrak{q} \in V(J)$が存在して,
        \[
            f^{\ast}(\mathfrak{q}) = f^{-1}(\mathfrak{q}) = \mathfrak{p}
        \]
        となる.
        今, $J \subseteq \mathfrak{q}$より, $f^{-1}(J) \subseteq f^{-1}(\mathfrak{q}) = \mathfrak{p}$なので,
        $\mathfrak{p} \in V(f^{-1}(J))$が成り立つ.
        あとは$V(f^{-1}(J)) \subseteq f^{\ast}(V(J))$を示せばよいが,
        これは$f^{\ast}$の制限によって$\Spec{B/J} \to \Spec{A/f^{-1}(J)}$が全射であることと同値である.

        まず, $f$がintegralであることと(5.6)より, $\iota: f(A) \to B$を包含射とすれば,
        $B/J$は$f(A)/\iota^{-1}(J)$上整である.
        また, $f(A)/\iota^{-1}(J) \subseteq B/J$とみれば, (5.10)より,
        \[
            \Spec{B/J} \to \Spec{f(A)/\iota^{-1}(J)}
        \]
        は全射である.
        ここで, $f(A) \cong A/\ker{f}$と$f^{-1}(\iota^{-1}(J)) = f^{-1}(J)$を考えれば,
        \[
            f(A)/\iota^{-1}(J) \cong (A/\ker{f})/(f^{-1}(J)/\ker{f}) \cong A/f^{-1}(J)
        \]
        となるので,
        \[
            \begin{tikzcd}
                A \arrow[r]\arrow[d] & B  \arrow[d] \\
                A/f^{-1}(J) \arrow[r] &  B/J
            \end{tikzcd}
        \]
        が可換であることから,
        \[
            \begin{tikzcd}
                \Spec{B/J} \arrow[r]\arrow[d] & \Spec{A/f^{-1}(J)}   \arrow[d] \\
                \Spec{B} \arrow[r] & \Spec{A}
            \end{tikzcd}
        \]
        も可換であることに注意すれば,
        $f^{\ast}$の制限によって$\Spec{B/J} \to \Spec{A/f^{-1}(J)}$が全射であることが従う.
    \end{problem}

    \begin{problem}
        $\mathfrak{p} = \ker{f}$とする.
        (5.10)より, ある$\mathfrak{q}$が存在して, $\mathfrak{p} = \mathfrak{q} \cap A$が成り立つ.
        このとき,
        \[
            \begin{tikzcd}
                A \arrow[r]\arrow[d] & B  \arrow[d] \\
                A/\mathfrak{p} \arrow[r, "\iota_0"] &  B/\mathfrak{q}
            \end{tikzcd}
        \]
        が可換であり, $\iota_0$が単射なので, $A/\mathfrak{p} \subseteq B/\mathfrak{q}$としてよい.
        (5.6)より, これは整拡大である.
        また, $A/\mathfrak{p}, B/\mathfrak{q}$がともに整域であることから,
        $K(A/\mathfrak{p}) \subseteq K(B/\mathfrak{q})$は代数的拡大になる.
        さらに, $f(A) \cong A/\ker{f}$であることから, $K(f(A)) \subseteq K(B/\mathfrak{q})$とみれば,
        $K(f(A)) \subseteq \Omega$でもあるので, $K(B) \subseteq \Omega$とみなせる.
    \end{problem}

    \begin{problem}
        $D \subseteq B' \otimes_A C$を$(f\otimes_A 1)(B \otimes C)$上整な元の集合とすれば,
        (5.3)より, $B'\otimes_A C$の部分環である.
        また, $B, C$が$A$-代数であることと, $f$が代数準同型であることから,
        $x \otimes y \in B'\otimes_A C$が$(f\otimes_A 1)(B \otimes C)$
        上整であることを示せば, 十分である.
        ここで, $x$は$f(B)$上整なので, $b^n = 1$とすれば, ある$b_i \in B'$が存在して,
        \[
            \sum_{i = 0}^n b_ix^i = 0
        \]
        となる.
        このとき,
        \[
            \sum_{i = 0}^n (b_i \otimes y^{n-i})(x\otimes y) = \sum_{i=0}^n ((b_ix)\otimes y^n) = \qty(\sum_{i = 0}^n b_ix^i)\otimes y^n = 0
        \]
        となるので, $x\otimes y \in B'\otimes_A C$は$B\otimes_A C$上整である.
    \end{problem}

    \begin{problem}
        $A = k[x^2-1], B = k[x], \mathfrak{n} = (x-1)B$とする.
        このとき,
        \[
           \mathfrak{m} =  A \cap (x-1)B = (x^2-1)A
        \]
        となることに注意する.

        $1/(x+1)$が$A_{\mathfrak{m}}$上整と仮定する.
        このとき, $a_0 = 1$となるようなある$a_i \in A$と$s_i \in A \setminus \mathfrak{m}$が存在して,
        \[
            \sum_{i = 0}^n a_is_i(x-1)^i = 0
        \]
        が成り立つ.
        このとき, $s_0 \in (x-1) \cap A = \mathfrak{m}$となるが, これは矛盾.
    \end{problem}

    \begin{problem}
        \begin{enumerate}
            \item  $x$の$B$における逆元を$y \in B$とする.
            $B$は$A$上整なので, $a_i \in A$が存在して,
            \[
                y^n + \cdots + a_{n-1}y + a_n = 0
            \]
            となるが, $n = 1$のときには$y = a_1 \in A$なので, 成り立つ.
            次に, $n-1$のときには, 主張が成り立つと仮定する.
            \[
                -xa_n = y^{n-1} + \cdots + a_{n-1}
            \]
            なので, $a_{n-1}$を$-xa_n$でおきなおせば,
            \[
                y^{n-1} + \cdots + a_{n_1} = 0
            \]
            となる.
            帰納法の仮定より, $y \in A$が成り立つ.
            ゆえに, $x$は$A$においても単元.
            \item (5.10)より, 縮約による全射$\Spec{B} \to \Spec{A}$が存在するが,
            特に縮約による全射$\Max{B} \to \Max{A}$も存在するので, 共通部分の逆像が逆像の共通部分であることから, 主張が成り立つ.
        \end{enumerate}
    \end{problem}   
\end{document}
