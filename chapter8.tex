\documentclass[dvipdfmx]{jsarticle}

\usepackage{amsthm}
\usepackage{enumerate}
\usepackage{amsmath}
\usepackage{amssymb}
\usepackage{cleveref}
\usepackage{physics}
\usepackage{color}
\usepackage{tikz-cd}

\newcounter{BaseCounter}[section]

\newtheoremstyle{JapanesePropositionStyle}{24pt}{}{}{}{\bfseries}{.}{1em}{\thmname{#1}\hspace{0.2em}\thmnumber{#2}\thmnote{\hspace{0.5em}(#3)}}
\theoremstyle{JapanesePropositionStyle}

\newtheorem{proposition}[BaseCounter]{Proposition}
\newtheorem{problem}[BaseCounter]{問題}
\newtheorem*{note}{Note}
\newtheorem{lemma}{補題}
\newtheorem{corollary}[BaseCounter]{Corollary}
\newtheorem{theorem}[BaseCounter]{Theorem}
\newtheorem{definition}[BaseCounter]{Definition}
\newtheorem{example}[BaseCounter]{Example}

\makeatletter
\newcommand\RedeclareMathOperator{%
  \@ifstar{\def\rmo@s{m}\rmo@redeclare}{\def\rmo@s{o}\rmo@redeclare}%
}
\newcommand\rmo@redeclare[2]{%
  \begingroup \escapechar\m@ne\xdef\@gtempa{{\string#1}}\endgroup
  \expandafter\@ifundefined\@gtempa
     {\@latex@error{\noexpand#1undefined}\@ehc}%
     \relax
  \expandafter\rmo@declmathop\rmo@s{#1}{#2}}
\newcommand\rmo@declmathop[3]{%
  \DeclareRobustCommand{#2}{\qopname\newmcodes@#1{#3}}%
}
\@onlypreamble\RedeclareMathOperator
\makeatother

\DeclareMathOperator{\ch}{char}
\DeclareMathOperator{\Hom}{Hom}
\DeclareMathOperator{\Emb}{Emb}
\DeclareMathOperator{\Aut}{Aut}
\DeclareMathOperator{\sign}{sign}
\DeclareMathOperator{\supp}{supp}
\DeclareMathOperator{\interior}{int}
\DeclareMathOperator{\relint}{relint}
\DeclareMathOperator{\Div}{Div}
\DeclareMathOperator{\Spec}{Spec}
\DeclareMathOperator{\Max}{Max}
\DeclareMathOperator{\coker}{coker}
\DeclareMathOperator{\Quot}{Quot}
\DeclareMathOperator{\Gal}{Gal}

\DeclareMathOperator*{\colim}{colim}

\RedeclareMathOperator{\Im}{Im}

\newcommand{\emb}{\hookrightarrow}

\newcommand{\Mod}[1]{\textbf{Mod}_{#1}}
\newcommand{\fMod}[1]{\textbf{fMod}_{#1}}
\newcommand{\QCoh}[1]{\textbf{QCoh}_{#1}}
\newcommand{\Coh}[1]{\textbf{Coh}_{#1}}
\newcommand{\set}[1]{\{#1\}}

\renewcommand{\theenumi}{\arabic{enumi}}
\renewcommand{\labelenumi}{(\theenumi)}


\begin{document}
    \stepcounter{BaseCounter}
    \begin{problem}
        \begin{description}
            \item[(i) $\Rightarrow$(ii)]
                (8.1)と(8.3)より, $A$の素イデアルは有限個であり, すべて極大イデアルである.
                ゆえに, $\Spec{A}$はすべての点が閉集合な有限集合なので, 離散空間である.
            \item[(ii) $\Rightarrow$ (iii)]
                明らか.
            \item[(iii) $\Rightarrow$ (i)]
                $\Spec{A}$の任意の一点集合は閉なので, $A$の任意の素イデアルは極大イデアルである.
                ゆえに, $\dim{A} = 0$であり, $A$はNoetherianなので, (8.5)より, (i)が成り立つ.
        \end{description}
    \end{problem}

    \begin{note}
        $\dima{A} = 0$かつNotherianでない例は$\mathbb{Q}[X_1, \dots, ]/(X_iX_j)$.
    \end{note}

    \begin{problem}
        \begin{description}
            \item[(i) $\Rightarrow$ (ii)] (8.7)より, Artin環は有限個のArtin局所環R$A_i$の直積で表せる.
            $A$が有限生成$k$代数であることから, $A_i$も有限生成$k$代数である.
            $A_i$の極大イデアルを$\mathfrak{m}_i$とすれば, $A_i/\mathfrak{m}_i$
            は有限生成$k$代数であって, さらに体なので, (5.24)より,
            $A_i/\mathfrak{m}_i$は$k$の有限次拡大である.
            このとき, 準同型$k \to A_i \to \End(A_i/\mathfrak{m}_i)$が存在するので, $k$線形空間であり,
            $A_i/\mathfrak{m}_i$は有限次元$k$線形空間となる.
            さらに, $A_i$は特にArtin環なので, \color{red}7.18\color{black}より,
            $A_i$加群として有限の長さをもつ.
            つまり, 次のようなイデアル$I_{i,j} \subseteq A_i$イデアルの列
            \[
                0 = I_{i,0} \subseteq I_{i,1} \subseteq \cdots \subseteq I_{i,n_i} = A
            \]
            であって, $I_{i,j+1}/I_{i,j} \cong A_i/\mathfrak{p}$を満たすものが存在する.
            ただし, $\mathfrak{p} \in \Spec{A_i}$である.
            ここで, $A_i$がArtin局所環であることから, $\Spec{A_i}  = \set{\mathfrak{m}_i}$となるので,
            $I_{i, j+1}/I_{i,j} = A_i/\mathfrak{m_i}$である.
            また, $I_{i,j}$は$k$線形空間なので, 上の列は$k$線形空間としての組成列である.
            したがって, $A_i$は$k$線形空間としての組成列をもつ.
            これと, 6.10より, $A_i$は有限次元$k$線形空間である.
            最後に, 有限次元$k$線形空間の直積は有限次元$k$線形空間なので, $A$も有限次元$k$線形空間である.
            \item [(ii) $\Rightarrow$ (i)]
            $A$は特に有限次元$k$線形空間なので, 特に$k$線形空間として有限の長さをもつ.
            さらに, $A$のイデアルが特に$k$線形空間であることに注意すれば, $A$は降鎖条件を満たすので, $A$はArtin環である.
         \end{description}
    \end{problem}

    \begin{problem}
        \begin{description}
            \item[(i) $\Rightarrow$ (iii)]
                $B$が有限生成$A$加群なので, $B = \sum_{i = 1}^n Ab_i$と表せる.
                任意の$k\otimes b \in k(\mathfrak{p}) \otimes_A B$に対して,
                ある$a_i \in A$が存在して,
                \begin{align*}
                    k \otimes  b = k\otimes \sum_{i = 1}^n a_ib_i = \sum_{i = 1}^n k\otimes a_ib_i = \sum_{i = 1}^n a_ik \otimes b_i = \sum_{i = 1}^n a_ik(1\otimes b_i)
                \end{align*}
                となるので, $k\otimes_A B$は$k(\mathfrak{p})$加群として, $1\otimes b_i$で生成される.
                したがって, $k \otimes_A B$は有限$k(\mathfrak{p})$代数である.

            \item [(iii) $\Rightarrow$ (ii)]
                8.3より, $k(\mathfrak{p})\otimes_A B$はArtin環である.
                また, 8.2より, $\Spec(k(\mathfrak{p}) \otimes_A B)$は離散空間であり,
                3.21より, ${f^\ast}^{-1}(\mathfrak{p}) = \Spec{(k(\mathfrak{p}) \otimes_A B)}$
                となるので, 任意の$\mathfrak{p} \in \Spec{A}$に対して, $f^\ast$の$\mathfrak{p}$上
                のファイバーは$\Spec{B}$の離散部分空間である.
            \item [(iii) $\Rightarrow$ (iv)]
                (iii) $\Rightarrow$ (ii)と全く同様に成り立つ.
            \item [(ii) $\Rightarrow$ (iii)]
                $k(\mathfrak{p}) \otimes_A B$は有限生成$k(\mathfrak{p})$代数である.
                実際, $B$が有限生成$A$代数であることから, 全射$A[X_1, \dots, X_n] \to B$が存在し, テンソル積は
                全射を保存するので,
                \[
                    k(\mathfrak{p})[X_1, \dots, X_n] = k(\mathfrak{p}) \otimes_A (A[X_1, \dots, X_n]) \longrightarrow k(\mathfrak{p}) \otimes_A B
                \]
                なる全射が存在する.
                また, $k(\mathfrak{p})$がNotherianなので, (7.7)より, $B\otimes_A k(\mathfrak{p})$もNotherianになり,
                8.2より, $k(\mathfrak{p})\otimes_A B$はArtin環である.
                さらに, 8.3より, $k(\mathfrak{p}) \otimes_A B$は有限$k(\mathfrak{p})$は有限$k(\mathfrak{p})$代数である.
        \end{description}

        $A = \mathfrak{Q}$, $\mathfrak{Q}[x_1, \dots]$ ($x_i$は代数的数)とすれば, 反例になる.
    \end{problem}


    \begin{problem}
        \color{red}
            有界の定義が謎ですねえ。
        \color{black}
    \end{problem}


    \begin{problem}
        局所化と剰余について, Noether性は保存され, イデアルの対応関係が存在することと,
        準素イデアルの準同型による逆像は準素イデアルなので,
        $B = A_\mathfrak{p}/\mathfrak{q}A_\mathfrak{p}$
        について考える.
        $\mathfrak{p}' \in \Spec{B}$とすれば, $A$において,
        \[
            \mathfrak{p}' = \sqrt{\mathfrak{p}'} \supseteq \sqrt{\mathfrak{q}} = \mathfrak{p}
        \]
        が成り立つ.
        これを$B$において考えれば, $\Spec{B} = \set{\mathfrak{p}}$が成り立つ.
        ゆえに, $\dim{B} = 0$であり, 特に$B$はArtin環である.
        さらに, 一般の真なるイデアル$I\subseteq B$についても, 同様にして, $\sqrt{I} = \mathfrak{p}$が成り立つ.
        つまり, $B$の任意のイデアルは$\mathfrak{p}$-準素イデアルである.
        $B$がArtinであることから, $B$の組成列は有限である.
        これと, 準素イデアルの対応関係が存在することから, $\mathfrak{q}$から$\mathfrak{p}$への準素イデアルの鎖は有限かつ有界である.
    \end{problem}

    \begin{note}
        $A = k[X_1, \dots, ]$とすれば,
        \[
            0 \subsetneq (X_1) \subsetneq (X_1, X_2) \subsetneq \cdots
        \]
        となるので, 長さが無限の素イデアルの列をもつ.
    \end{note}

\end{document}
