\documentclass[dvipdfmx]{jsarticle}

\usepackage{amsthm}
\usepackage{enumerate}
\usepackage{amsmath}
\usepackage{amssymb}
\usepackage{cleveref}
\usepackage{physics}
\usepackage{color}
\usepackage{tikz-cd}

\newcounter{BaseCounter}[section]

\newtheoremstyle{JapanesePropositionStyle}{24pt}{}{}{}{\bfseries}{.}{1em}{\thmname{#1}\hspace{0.2em}\thmnumber{#2}\thmnote{\hspace{0.5em}(#3)}}
\theoremstyle{JapanesePropositionStyle}

\newtheorem{proposition}[BaseCounter]{Proposition}
\newtheorem{problem}[BaseCounter]{Problem}
\newtheorem*{note}{Note}
\newtheorem{lemma}[BaseCounter]{Lemma}
\newtheorem{corollary}[BaseCounter]{Corollary}
\newtheorem{theorem}[BaseCounter]{Theorem}
\newtheorem{definition}[BaseCounter]{Definition}
\newtheorem{example}[BaseCounter]{Example}

\makeatletter
\newcommand\RedeclareMathOperator{%
  \@ifstar{\def\rmo@s{m}\rmo@redeclare}{\def\rmo@s{o}\rmo@redeclare}%
}
\newcommand\rmo@redeclare[2]{%
  \begingroup \escapechar\m@ne\xdef\@gtempa{{\string#1}}\endgroup
  \expandafter\@ifundefined\@gtempa
     {\@latex@error{\noexpand#1undefined}\@ehc}%
     \relax
  \expandafter\rmo@declmathop\rmo@s{#1}{#2}}
\newcommand\rmo@declmathop[3]{%
  \DeclareRobustCommand{#2}{\qopname\newmcodes@#1{#3}}%
}
\@onlypreamble\RedeclareMathOperator
\makeatother

\DeclareMathOperator{\ch}{char}
\DeclareMathOperator{\Hom}{Hom}
\DeclareMathOperator{\Emb}{Emb}
\DeclareMathOperator{\Aut}{Aut}
\DeclareMathOperator{\sign}{sign}
\DeclareMathOperator{\supp}{supp}
\DeclareMathOperator{\interior}{int}
\DeclareMathOperator{\relint}{relint}
\DeclareMathOperator{\Div}{Div}
\DeclareMathOperator{\Spec}{Spec}
\DeclareMathOperator{\Max}{Max}
\DeclareMathOperator{\coker}{coker}
\DeclareMathOperator{\Quot}{Quot}

\DeclareMathOperator*{\colim}{colim}

\RedeclareMathOperator{\Im}{Im}

\newcommand{\emb}{\hookrightarrow}

\newcommand{\Mod}[1]{\textbf{Mod}_{#1}}
\newcommand{\fMod}[1]{\textbf{fMod}_{#1}}
\newcommand{\QCoh}[1]{\textbf{QCoh}_{#1}}
\newcommand{\Coh}[1]{\textbf{Coh}_{#1}}

\renewcommand{\theenumi}{\arabic{enumi}}
\renewcommand{\labelenumi}{(\theenumi)}


\begin{document}
    \setcounter{BaseCounter}{17}
    \begin{problem}
        \begin{description}
            \item[(i) $\Rightarrow$ (ii)]
                (3.6)より, すぐに従う.
            \item[(ii) $\Rightarrow$ (iii)]
                任意の$\mathfrak{m}$に対して, ある$\mathfrak{n} \in \Spec{B}$が存在して,
                $f^\ast: \Spec{B} \to \Spec{A}$とすれば,
                $f^\ast(\mathfrak{n}) = \mathfrak{m}$となる.
                また, $f$が全射であることから,
                \[
                    \mathfrak{m}^e = f(\mathfrak{m})B = f(f^\ast(\mathfrak{m}))B = f(f^{-1}(\mathfrak{m}))B = \mathfrak{n}B = \mathfrak{n}
                \]
                となるので, $\mathfrak{m} \neq (1)$が成り立つ.
            \item[(iii) $\Rightarrow$ (iv)]
                $x \in M \setminus \set{0}$とする.
                $M' = Ax$とすれば,
                \[
                    0 \longrightarrow M' \longrightarrow M
                \]
                が完全列であって, $B$が平坦であることから,
                \[
                    0 \longrightarrow M' \otimes_A B \longrightarrow M \otimes_A B
                \]
                も完全となる.
                ゆえに, $M' \otimes_A B \neq 0$を示せば十分である.
                $f: A \to Ax$を自然なものとすれば, $A/\ker{f} \cong M'$となる.
                ここで, $\ker{f} \neq A$より, ある$\mathfrak{m} \in \Max{A}$が存在して,
                $A \to B$から$A/\mathfrak{m} \to B/\mathfrak{m}^e$が導かれる.
                このとき,
                \[
                    M' \otimes_A B = A/\mathfrak{m} \otimes_A B = B/\mathfrak{m}B = B/\mathfrak{m}^e
                \]
                となる.
                (iii)より, $\mathfrak{m}^c \neq (1)$なので,
                $M'_B = M' \otimes_A B \neq 0$が成り立つ.
            \item[(iv) $\Rightarrow$ (v)]
            $f: M \ni x \to 1 \otimes x \in M_B$として, $M' := \ker{f}$とすれば,
            \[
                0 \longrightarrow M' \longrightarrow M \longrightarrow M_B
            \]
            が完全になる.
            $B$が平坦であることから,
            \[
                0 \longrightarrow M' \otimes_A B \longrightarrow M \otimes_A B \longrightarrow M_B \otimes_A B
            \]
            は完全列である.
            さらに, $M \otimes_A B$は$B$-加群であって, 射による元の対応も2.13の条件を満たすので,
            $M_B \to M_B \otimes_A B$は単射である.
            ゆえに, $M'\otimes_A B = 0$となるが, (iv)より, $M' = \ker{f} = 0$となるので, $f$は単射である.
        \item[(v) $\Rightarrow$ (i)]
            $\mathfrak{a}$を$A$の任意のイデアルとする.
            $M = A/\mathfrak{a}$とすれば, (v)より, $A/\mathfrak{a} = M \to M \otimes_A B = B/\mathfrak{a}^e$は単射なので,
            $\mathfrak{a}^{ec} \subseteq \mathfrak{a}$が成り立つ.
            逆は(1.17)より, 自明に成り立つ.
        \end{description}
    \end{problem}
    \begin{problem}
        $M \to N$を単射とする.
        $g$は忠実平坦なので,
        \[
            \begin{tikzcd}
                0 \arrow[r] & M_B  \arrow[d] \arrow[r] & M_B \otimes_B C \arrow[r]\arrow[d] & \coker(M_B \to M_B \otimes_B C) \arrow[d] \arrow[r] & 0 \\
                0 \arrow[r] &  N_B \arrow[r] & N_B \otimes_B C\arrow[r] & \coker(N_B \to N_B \otimes_B C) \arrow[r] & 0
            \end{tikzcd}
        \]
        は各行が完全な可換図式である.
        また, $M_B \otimes_B C = M \otimes_A C$であることと, $g\circ f$が平坦であることから,
        $M_B \otimes_B C \to N_B \otimes_B C$は単射である.
        ゆえに, 上の図式に対して蛇の補題を使えば, $0 \to M_B \to N_B$は完全になり,
        これは$f$が平坦であることを示している.
    \end{problem}
    \begin{problem}
        (3.10)より, $B_\mathfrak{p}$は平坦$A_\mathfrak{p}$加群であって,
        $f$から自然に環の射$A_\mathfrak{p} \ni a/s \mapsto f(a)/s \in B_\mathfrak{p}$が得られる.
        また, $B_\mathfrak{q}$は$B_\mathfrak{p}$の局所化なので, 3.3より, 平坦$A_\mathfrak{p}$加群であり,
        自然な射$B_\mathfrak{p} \ni b/s \mapsto b/s \in B_\mathfrak{q}$が存在するので,
        環の射$A_\mathfrak{p} \to B_\mathfrak{q}$が得られる.
        この射により, 次の図式は可換図式になる.
        \[
            \begin{tikzcd}
                A_\mathfrak{p} \arrow[r]\arrow[d] & B_\mathfrak{q}  \arrow[d] \\
                A \arrow[r] &  B
            \end{tikzcd}
        \]
        したがって,
        \[
            \begin{tikzcd}
                \Spec{A_\mathfrak{p}}  & \arrow[l]\Spec{B_\mathfrak{q}} \\
                \Spec{A} \arrow[u] &  \arrow[l]\Spec{B} \arrow[u]
            \end{tikzcd}
        \]
        も可換であり, これより, $\Spec{A_\mathfrak{p}}\to \Spec{B_\mathfrak{q}}$が
        $f^\ast$の制限として得られる.
        あとは, $g:A_\mathfrak{p} \to B_\mathfrak{q}$が忠実平坦であることを示せば十分である.
        $A_\mathfrak{p}$の極大イデアルは$\mathfrak{p}A_\mathfrak{p}$であって,
        $\mathfrak{p} = f^{-1}(\mathfrak{q})$なので,
        $g(\mathfrak{p}A_\mathfrak{p}) \subseteq \mathfrak{q}B_\mathfrak{q}$が成り立つ.
        したがって, $g$は忠実平坦であり,3.16より,
        $\Spec{B_\mathfrak{q}} \to \Spec{A_\mathfrak{p}}$は全射になる.
    \end{problem}
\end{document}
